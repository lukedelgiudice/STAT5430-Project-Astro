\documentclass[11pt,]{article}
\usepackage[left=1in,top=1in,right=1in,bottom=1in]{geometry}
\newcommand*{\authorfont}{\fontfamily{phv}\selectfont}
\usepackage[]{libertine}


  \usepackage[T1]{fontenc}
  \usepackage[utf8]{inputenc}




\usepackage{abstract}
\renewcommand{\abstractname}{}    % clear the title
\renewcommand{\absnamepos}{empty} % originally center

\renewenvironment{abstract}
 {{%
    \setlength{\leftmargin}{0mm}
    \setlength{\rightmargin}{\leftmargin}%
  }%
  \relax}
 {\endlist}

\makeatletter
\def\@maketitle{%
  \newpage
%  \null
%  \vskip 2em%
%  \begin{center}%
  \let \footnote \thanks
    {\fontsize{18}{20}\selectfont\raggedright  \setlength{\parindent}{0pt} \@title \par}%
}
%\fi
\makeatother




\setcounter{secnumdepth}{0}

\usepackage{longtable,booktabs}



\title{A Data-Driven Framework for Game Development \thanks{Formatting
based on svmille guide (\url{http://github.com/svmiller}).}  }
 



\author{\Large Luke Del Giudice
(wcc5ub)\vspace{0.05in} \newline\normalsize\emph{University of
Virginia}  }


\date{}

\usepackage{titlesec}

\titleformat*{\section}{\normalsize\bfseries}
\titleformat*{\subsection}{\normalsize\itshape}
\titleformat*{\subsubsection}{\normalsize\itshape}
\titleformat*{\paragraph}{\normalsize\itshape}
\titleformat*{\subparagraph}{\normalsize\itshape}


\usepackage{natbib}
\bibliographystyle{apsr}
\usepackage[strings]{underscore} % protect underscores in most circumstances



\newtheorem{hypothesis}{Hypothesis}
\usepackage{setspace}


% set default figure placement to htbp
\makeatletter
\def\fps@figure{htbp}
\makeatother

\usepackage{hyperref}
\usepackage{array}
\usepackage{caption}
\usepackage{graphicx}
\usepackage{siunitx}
\usepackage[table]{xcolor}
\usepackage{multirow}
\usepackage{hhline}
\usepackage{calc}
\usepackage{tabularx}
\usepackage{fontawesome}
\usepackage[para,online,flushleft]{threeparttable}
\usepackage{tcolorbox}
\newtcolorbox{definitionbox}{ colback=gray!10, colframe=black!60, boxrule=0.5pt, arc=2pt, left=6pt, right=6pt, top=4pt, bottom=4pt, title=Definition }

% move the hyperref stuff down here, after header-includes, to allow for - \usepackage{hyperref}

\makeatletter
\@ifpackageloaded{hyperref}{}{%
\ifxetex
  \PassOptionsToPackage{hyphens}{url}\usepackage[setpagesize=false, % page size defined by xetex
              unicode=false, % unicode breaks when used with xetex
              xetex]{hyperref}
\else
  \PassOptionsToPackage{hyphens}{url}\usepackage[draft,unicode=true]{hyperref}
\fi
}

\@ifpackageloaded{color}{
    \PassOptionsToPackage{usenames,dvipsnames}{color}
}{%
    \usepackage[usenames,dvipsnames]{color}
}
\makeatother
\hypersetup{breaklinks=true,
            bookmarks=true,
            pdfauthor={Luke Del Giudice (wcc5ub) (University of
Virginia)},
             pdfkeywords = {},  
            pdftitle={A Data-Driven Framework for Game Development},
            colorlinks=true,
            citecolor=blue,
            urlcolor=blue,
            linkcolor=magenta,
            pdfborder={0 0 0}}
\urlstyle{same}  % don't use monospace font for urls

% Add an option for endnotes. -----


% add tightlist ----------
\providecommand{\tightlist}{%
\setlength{\itemsep}{0pt}\setlength{\parskip}{0pt}}

% add some other packages ----------

% \usepackage{multicol}
% This should regulate where figures float
% See: https://tex.stackexchange.com/questions/2275/keeping-tables-figures-close-to-where-they-are-mentioned
\usepackage[section]{placeins}


\begin{document}
	
% \pagenumbering{arabic}% resets `page` counter to 1 
%    

% \maketitle

{% \usefont{T1}{pnc}{m}{n}
\setlength{\parindent}{0pt}
\thispagestyle{plain}
{\fontsize{18}{20}\selectfont\raggedright 
\maketitle  % title \par  

}

{
   \vskip 13.5pt\relax \normalsize\fontsize{11}{12} 
\textbf{\authorfont Luke Del Giudice
(wcc5ub)} \hskip 15pt \emph{\small University of Virginia}   

}

}








\begin{abstract}

    \hbox{\vrule height .2pt width 39.14pc}

    \vskip 8.5pt % \small 

\noindent I have been co-developing a first-person shooter video game
with my brother since the beginning of college, almost four years ago.
The game, called
\href{https://store.steampowered.com/app/2540620/Astro/}{Astro}, can be
summarized as ``Call of Duty in zero gravity.'' Multiplayer shooters are
typically produced by large studios, and balancing such games (tuning
weapons, movement, and other systems in a high-dimensional design space)
usually requires entire teams dedicated to the task. Since Astro's
development team consists only of my brother and me, I have not been
able to invest significant time into systematic balancing or data-driven
decision making. This changed recently with a redesign of Astro's
networking architecture to be fully deterministic and driven entirely by
player inputs, which I can store efficiently. As a result, entire
matches can now be resimulated and reviewed without storing the full
game state, unlike traditional networking solutions. This capability has
enabled several development improvements, including the ability to
collect data easily after games by resimulating matches and extracting
all relevant information from the reconstructed game state. In this
project, I collected and analyzed data from the most recent Astro
playtest to provide insights for future balancing and development,
ensuring that upcoming changes reflect actual gameplay rather than our
limited personal experience or the subjective impressions captured in
post-playtest surveys.


    \hbox{\vrule height .2pt width 39.14pc}


\end{abstract}


\vskip -8.5pt


 % removetitleabstract

\noindent  

\section{Introduction}\label{introduction}

Astro's networking solution, \emph{NetTick}, is a fully deterministic
lockstep networking system, which runs the entire game simulation
detached from the presentation (graphics, sound, UI, etc).

\begin{definitionbox}
\textbf{Deterministic Lockstep:} Networking Model where only inputs are sent across the network, and the game is simulated completely deterministically so that the results on all clients is exactly the same, bit-for-bit, without state replication.
\end{definitionbox}

The result, as mentioned, is that any game can be replayed by simply
replaying the sequence of inputs, which just requires capturing the
input stream of every player. An input snapshot is 41 bytes, however, a
compressed sequence can get down to approximately 3 bytes per input.
Therefore, serializing an entire game (approximately 10 to 15 minutes
with 8 players) is just 200 to 400 KB. This makes it possible to upload
every game to an AWS S3 bucket. Later, when I want to run analyses, I
attach a custom measurement script to the game simulation and spin up a
cluster of EC2 instances. Each instance pulls replay files from the S3
bucket, re-simulates the corresponding games, and uses the script to
collect and aggregate the desired data in the background. This process
of re-simulation took approximately one and a half hours for the
analysis outline in this report on 113 games played by 119 different
playtesters. I have attached the C++ script
\emph{ReplayDriver\_GameStats.cpp} that performed data aggregation in
the submitted repo, and if it's needed for confirmation I can send the
executable that actually performs the simulation like usual after
attaching \emph{UReplayDriver\_GameStats} as the replay driver and
writes the results to disk. However, while this replay and initial
aggregation pipeline is relevant to a computing course, its remaining
implementation details are largely domain-specific (custom JSON
serialization and interactions with \emph{NetTick} specific
architecture, which is proprietary and outside the scope of this work),
and the overall AWS workflow closely mirrors the structure of the
subsequent analysis stage. The more interesting component is the offline
analysis of the aggregated data, which is where the statistical
techniques come into play. Like the replay step, this analysis is
embarrassingly parallel, so the multiprocessing design is
straightforward; most of the effort went into specifying and
implementing the logic for the different proposed studies.

These studies address a range of key questions that I developed to align
with our current development priorities. This report outlines the
overall code structure of the repo, explains the three data structures
involved, discusses key challenges and requirements of processing the
data, and finally describes the proposed studies along with their
implementations and results.

\section{Approximate Timeline}\label{approximate-timeline}

I initially spent time reading balancing papers to understand how larger
studios and academics approached data-driven balancing. While this did
inspire some of the studies I created, most of the papers were academic
and relied on synthetic data, which is entirely inappropriate for
balancing a high-skill 3D shooter. After this research phase and
extensive discussion with my brother, I began
\href{https://docs.google.com/document/d/1m_yEYKeCD22brKKyuOgPoGG0nfoaZXBNX59krkBNRSg/edit?usp=sharing}{writing
up} the actual studies we wanted to pursue. Finally, I implemented the
studies that were deemed highest priority and not focused on long-term
longitudinal results.

\section{Overview of the Pipeline}\label{overview-of-the-pipeline}

The data flows through two distinct phases: extraction (per match) and
analysis (per study). The \emph{master\_analysis.py} script acts as the
central hub, passing data objects between the \textbf{etl} layer and the
\textbf{analysis} layer.

To be more specific: the script \emph{game\_etl\_core.py} is where all
of the logic processing game state lives, \emph{game\_etl.py} acts as an
interface between this game state logic and the
\emph{master\_analysis.py} script, all of the study files are in the
\emph{studies/} directory, and the results of running
\emph{master\_analysis.py} are stored in the \emph{results/} folder
(results are further organized into the subfolders \emph{graphs/},
\emph{models/}, \emph{spatial/}, and \emph{data/}).

\section{Multiprocessing and
Optimization}\label{multiprocessing-and-optimization}

Given that this analysis is going to be used for future (hopefully)
larger playtests, I decided to utilize Python's
\emph{concurrent.futures.ProcessPoolExecutor} to implement
multiprocessing (this was chosen over multithreading due to Python's GIL
limiting true parallelism for CPU-bound tasks).

\section{(1) Data Parallelism}\label{data-parallelism}

In the first phase, the workload is embarrassingly parallel. Each match
(consisting of one JSON file and two CSVs) is independent of every other
match.

\begin{itemize}
\tightlist
\item
  I first detect the number of available logical cores.
\item
  Then I map the list of 113 match file groups across these cores.
\item
  Then each worker process instantiates its own \emph{GameProcessor} and
  executes the etl logic for all studies.
\item
  Lastly, the resulting data dictionaries are serialized (pickled) and
  returned to the main process.
\end{itemize}

\section{(2) Task Parallelism}\label{task-parallelism}

In the second phase, I pivoted to task-based parallelism. Once all data
is aggregated into memory, the independent studies are submitted as
separate tasks.

\begin{itemize}
\tightlist
\item
  To prevent pickling errors common with Python modules, a dynamic
  import wrapper (\emph{run\_study\_wrapper}) is used. The master script
  passes the string name of the module to the worker.
\item
  The worker dynamically imports the module and executes the \emph{run}
  function.
\item
  This ensures that CPU-intensive tasks, such as
  \hyperref[s8-skill-estimation]{s8} or
  \hyperref[s9-playstyle-archetypes]{s9}, do not block the generation of
  simpler studies (ie:
  \hyperref[s0s1-metadata-and-descriptive-statistics]{s0}).
\end{itemize}

\section{(3) Performance Benchmarks}\label{performance-benchmarks}

\begin{longtable}[]{@{}lccc@{}}
\caption{Performance Impact of Parallel Processing (24 CPU
cores)}\tabularnewline
\toprule\noalign{}
Metric & SingleCore & MultiCore & ImprovementFactor \\
\midrule\noalign{}
\endfirsthead
\toprule\noalign{}
Metric & SingleCore & MultiCore & ImprovementFactor \\
\midrule\noalign{}
\endhead
\bottomrule\noalign{}
\endlastfoot
\textbf{Extraction Time} & 47s & 11s & \textbf{4.3X} \\
\textbf{Analysis Time} & 15s & 15s & \textbf{1X} (overhead) \\
\textbf{Total Runtime} & 1m 2s & 27s & \textbf{2.3X} \\
\end{longtable}

\section{Data Sources}\label{data-sources}

\section{(1) Match JSON}\label{match-json}

This is the complete record of all relevant events that happen in a
match. While any given game has tens of thousands of ticks, I can ignore
this complexity by mainly interacting with the relevant ticks as
determined by the previous re-simulation step.

\subsubsection{Key Content:}\label{key-content}

\begin{itemize}
\tightlist
\item
  \texttt{GameId}, \texttt{GameMode}, \texttt{Map},
  \texttt{DurationSec}, \texttt{winner} / \texttt{winning\ team} - basic
  metadata
\item
  \texttt{DeviceProfiles} -- array of hardware and graphics settings for
  every player in the match (GPU, CPU, resolution, FPS cap, etc)
\item
  \texttt{Players} - per-player summary statistics (Kills, Deaths,
  AvgSpeed, etc)
\item
  \texttt{Items} -- item usage statistics (PlayTime, Kills, Deaths, etc
  for each item)
\item
  \texttt{events} -- chronological event stream (the following list is
  not comprehensive)

  \begin{itemize}
  \tightlist
  \item
    \texttt{join}, \texttt{leave}, \texttt{spawn}, \texttt{set\_team},
    \texttt{set\_loadout}, \texttt{equip} - non-gameplay events
  \item
    \texttt{fire}, \texttt{throw}, \texttt{swing}, \texttt{stab},
    \texttt{burst} - various fight related events
  \item
    \texttt{damage} - whenever a player takes any damage
  \item
    \texttt{elim} - whenever a player dies
  \item
    \texttt{dash}, \texttt{kick}, \texttt{surface\_lock},
    \texttt{push\_off} - movement events
  \end{itemize}
\end{itemize}

\section{(2) Player Update CSV}\label{player-update-csv}

High-frequency telemetry snapshot of every player's state for when
required information is disconnected from events.

\subsubsection{Key Content:}\label{key-content-1}

\begin{itemize}
\tightlist
\item
  \texttt{Stamp} -- timestamp in the same units as the JSON events
  (\(1\) stamp = \(\frac{1}{40}\) second)
\item
  \texttt{PlayerId} -- the same ID that appears in the JSON events
\item
  \texttt{Health}, \texttt{Location.X}, \texttt{Location.Y},
  \texttt{Location.Z}
\item
  \texttt{Velocity.X}, \texttt{Velocity.Y}, \texttt{Velocity.Z}
\item
  and many other metrics relating to player state (\texttt{Ammo},
  \texttt{Rotation.Roll}, etc)
\end{itemize}

\section{(3) Performance CSV}\label{performance-csv}

Client performance collected for every player in the match for tracking
optimization.

\subsubsection{Key Content:}\label{key-content-2}

\begin{itemize}
\tightlist
\item
  \texttt{Frame\ Avg} - frame time in milliseconds
\item
  \texttt{Packet\ Latency} -- round-trip ping in milliseconds
\item
  \texttt{GPU\ Bound}, \texttt{GT\ Bound}, \texttt{NTT\ Bound} -
  percentage of frames that were bottlenecked by GPU, game thread, or
  \emph{NetTick} thread
\item
  and many other metrics relating to player performance or hardware
  (\texttt{RAM\ MB}, \texttt{GPU\ Avg}, etc)
\end{itemize}

\section{Methods and Implementations}\label{methods-and-implementations}

Here are the specific details of the data processing and studies
mentioned, and a high-level overview of their implementations. The ``P''
prefix means the section references processing, while ``S'' indicates a
corresponding study that matches the file prefixes in the repo.

\section{(P1) Tracking User State}\label{p1-tracking-user-state}

The match data processed is entirely event driven as previously
described. These events contain fields that further identify what's
happening in the event beyond what type it is. For example, an
\texttt{equip} event will also have the fields \texttt{stamp},
\texttt{player}, and \texttt{item}. When this event is examined it shows
that the specified player switched to that item at the given time.
However, a few complications arise from this examination. First, the
\texttt{player} field, which gives a player id integer, is not unique to
a single player. For example, a player could join and get assigned that
id, but if they disconnect mid-match then this id is free to be assigned
to a new player joining later. This is obviously a problem for analysis
that focuses on each {[}player's abilities{]}{[}(10) Skill
Estimation{]}, so I need to keep track of who's who using the
recoverable Steam usernames.

To solve this, the \emph{game\_etl.py} script reconstructs the timeline
in the \textbf{\_reconstruct\_gamestate} function where the mapping
between an id and player is a function of time. Every time a
\texttt{Join} event is detected, a new session segment for that specific
id is created linking it to the username. This segment remains open
until a corresponding \texttt{leave} event is found or the match ends.
When a specific data point needs to be processed, I simply resolve the
given id at the specified timestamp with the \textbf{resolve\_user}
function. This same logic is also used for tracking the loadout state of
a player. The script builds a parallel timeline of \texttt{equip}
events. To determine what item was used in a fight, I query this
timeline with the \textbf{get\_weapon\_at\_time} function for the last
known item change prior to the fight's start time.

\section{(P2) Fight Detection}\label{p2-fight-detection}

Raw game data is a continuous stream of discrete events: a shot fired
here, damage taken there, a random kick. The concept of a ``Fight'' does
not exist in the raw data; it is a logic concept that must be derived
for the purposes of analyzing thousands of fights systematically.

The challenge here was to create a definition that makes sense for
fights. This is the final definition of a ``Fight'' I used:

\begin{itemize}
\tightlist
\item
  Any time a player damages another (this is an event), a potential
  fight is started between just these two players (all fights under this
  definition are one versus one).
\item
  I look back 3 seconds to find the first ``fight action''
  (\texttt{fire}, \texttt{kick}, \texttt{throw}, etc) since Astro uses a
  projectile system (not hitscan) and the initial damage is rarely the
  result of an action occurring at the same timestamp as the
  \texttt{damage} event. This effectively rewinds the start time of the
  fight to include the initial shots that might have missed, providing a
  more accurate measurement of the engagement's actual duration.
\item
  From here, a specified timer (configured as 5 seconds) begins counting
  down. If no other ``fight action'' occurs by the two players fighting,
  then the fight is considered over and it's logged as a non-fatal
  fight. Otherwise, if one of the players dies, then I immediately
  identify all active fights involving that victim. These fights are
  marked as fatal, and the survivor in each pairing is credited with a
  win (but that win is scaled based on how much damage they
  contributed).
\end{itemize}

This process also builds on the tracking of user state from the previous
data processing; for example, verification needs to be done to make sure
both players are still in the game and any id still belongs to an
original fight initiator.

\section{(S0/S1) Metadata and Descriptive
Statistics}\label{s0s1-metadata-and-descriptive-statistics}

Before performing advanced modeling, I established a comprehensive
baseline of the dataset by aggregating high-level metadata and
fundamental player/item statistics across all 113 processed matches.

Game mode and map frequencies were computed directly from the
\texttt{GameMode} and \texttt{Map} fields in the match JSON files. These
reflect player preferences, as the game allows voting between matches.

Player-level statistics (total kills, deaths, playtime, and wins) were
aggregated using the identity resolution system described in Section 2.
This system maps transient numeric player ids to persistent usernames
across rejoins and multiple matches, enabling the construction of
lifetime profiles for each participant.

Item usage statistics were derived from the event stream:

\begin{itemize}
\tightlist
\item
  Weapon swaps were counted via \texttt{equip} events.
\item
  Time held was calculated by differencing consecutive equip timestamps
  per player, with the final segment capped at match end time.
\item
  First-pick rates were determined by tracking the first observed
  loadout per player per match.
\end{itemize}

These metrics provide an unbiased view of item popularity and player
behavior before effectiveness analysis.

\section{(S2) Health Regeneration
Dynamics}\label{s2-health-regeneration-dynamics}

The game features passive health regeneration: a fixed delay after
taking damage, followed by periodic healing ticks. To quantify how often
players enter fights below full health, which directly lowers effective
TTK, I modeled the long-term health distribution using a Markov chain
Monte Carlo simulation.

From the fight detection system, I extracted two empirical
distributions: * Total damage taken per fight (representing health lost
in a single engagement). * Idle time between fights (time available for
regeneration).

I then simulated thousands of fight--rest cycles using the exact in-game
regeneration rules (\(5\) second initial delay, \(+1\) health every 2
seconds). For each simulated fight, I sampled damage and idle time
independently from the observed distributions, applied regeneration
logic, and recorded the resulting starting health for the next fight.

The stationary distribution of this Markov chain represents the
equilibrium probability of beginning a fight at any health value from 0
to 10. This provides a data-driven estimate of average combat health,
and can even be used to test new regeneration rules since those values
are parameters in the code.

\section{(S3) Item Win Rate}\label{s3-item-win-rate}

Raw win rates per item are heavily confounded by context (distance,
health, player skill) and kill stealing. To isolate true item
effectiveness, I fitted a weighted binomial generalized linear model
(logistic regression) predicting fight outcome (win/loss) from the
primary item used, controlling for starting distance and health.

To solve kill-stealing and multi-weapon fights, I implemented a
damage-proportional fractional credit system:

\begin{itemize}
\tightlist
\item
  If a player dealt all 10 damage needed to kill, their item receives
  full weight (1.0).
\item
  If they dealt only 3 damage, their item receives weight 0.3.
\item
  The losing player's items are weighted by their own damage
  contribution.
\end{itemize}

This ensures cheap finishers (ie: switching to Sword at the last moment
after enemy is already low health) are down-weighted, while sustained
performers are properly credited. The model outputs odds ratios relative
to a baseline item (Astro Rifle), interpretable as: ``holding the Spark
increases win probability by 42\% versus holding the Astro Rifle, all
else equal.''

A mixed-effects version was initially planned to control for individual
player skill but was abandoned due to library limitations with frequency
weights. Damage attribution was prioritized over skill adjustment.

\section{(S4) Spatial Death Density}\label{s4-spatial-death-density}

There was no way to get an understanding of where fights were taking
place on the map from some 2D visual, as evidence by the 2D player death
heatmaps generated in this section using kernal density estimation. To
address this, I developed a 3D spatial analysis pipeline to visualize
high-density death clusters directly within the game's native engine
(Unreal Engine 5).

The spatial extraction process aggregated \(x, y, z\) coordinates for
all \texttt{elim} events. Because raw point clouds of thousands of
deaths are visually noisy and difficult to interpret, I applied a
two-stage clustering algorithm to identify hotspots:

\begin{itemize}
\tightlist
\item
  \textbf{Micro-Clustering}: I first applied MiniBatchKMeans to group
  local neighbors into small micro-clusters (5 deaths per group)
\item
  \textbf{Macro-Merging}: I then fed these centroids into an
  AgglomerativeClustering algorithm with a distance threshold of 3
  meters. This recursively merged adjacent micro-clusters until
  spatially distinct blobs of activity remained.
\end{itemize}

This hybrid approach reduced thousands of raw data points into a
manageable set of weighted centroids, where the weight corresponds to
the total death count in that specific area.

To visualize these 3D centroids, I engineered a custom interoperability
pipeline between Python, Blender, and Unreal Engine:

\begin{itemize}
\tightlist
\item
  \textbf{Geometry Generation (Blender)}: the script
  \emph{death\_cluster\_importer.py} ingested the processed CSV data
  into Blender. Using Geometry Nodes, it instanced spherical meshes at
  each centroid coordinate, scaling the sphere radius logarithmically
  based on the death count to distinguish high kill areas.
\item
  \textbf{Data Baking}: a critical challenge was preserving the death
  count data during the export to a static mesh. The
  \emph{death\_cluster\_exporter.py} script solved this by converting
  the geometry instances into actual polygons and mathematically mapping
  the death count to a dynamic blue to red color gradient. This gradient
  was baked directly into the mesh's vertex colors, ensuring the data
  persisted without needing complex textures.
\item
  \textbf{Engine Integration (Unreal)}: the resulting FBX was imported
  into Unreal Engine. By applying a material that drives Emissive Color
  using Vertex Color data, I projected the historical death density
  directly onto the level geometry.
\end{itemize}

This study will allow my brother and I to fly through the map and
instantly identify choke points, spawn traps, and unfair lines of sight
that are invisible in 2D analysis.

\section{(S5) Sequential Pattern
Mining}\label{s5-sequential-pattern-mining}

Given the movement-heavy nature of the game, I investigated whether
specific sequences of locomotion actions predict success or failure.

For every elimination, I extracted the 10 second movement event stream
(ie: \texttt{dash}, \texttt{kick}, \texttt{surface\_lock}``) for both
killer and victim. These were converted into 2-grams, such as
\texttt{dash} to \texttt{kick}. Frequencies were computed separately for
successful and unsuccessful outcomes. By comparing these frequencies, I
can identify tactical patterns; for example, discovering that a
\texttt{dash} to \texttt{kick} sequence appears 30\% more often in
successful engagements than in fatal ones would help confirm that
learning base movement is improving player's performance.

\section{(S6) Functional Velocity
Analysis}\label{s6-functional-velocity-analysis}

Beyond the discrete actions themselves, the actual shape of a player's
movement speed before a kill is also predictive. Does accelerating
immediately before a fight improve survival, or is sustained speed more
important? To test this, I used pointwise logistic regression on
functional velocity curves.

All curves were time-aligned to \(t=0\) (moment of kill/death) and
left-padded with NaN where data was unavailable (pre-spawn). At each
time point, a separate logistic regression predicted outcome (kill vs
death) from instantaneous speed.

The resulting coefficient curve shows when speed is most predictive; for
example, a strong positive peak at approximately \(-2.8\) seconds would
indicate that high velocity specifically during the approach phase is a
powerful predictor of winning the engagement.

\section{(S7) Survival Analysis}\label{s7-survival-analysis}

Theoretical TTK ignores accuracy, movement, and positioning in favor of
a simple fomula (\(\frac{\text{dps}}{\text{health}}\)). I computed real
behavioral TTK using the Kaplan--Meier estimator on fight durations from
the detection system discussed previously.

Two curves were produced:

\begin{itemize}
\tightlist
\item
  All engagements (including timeouts); this measures average fight
  length.
\item
  Fatal fights only; this is the true empirical TTK distribution
\end{itemize}

The median of the fatal fights curve represents the time at which 50\%
of lethal fights conclude, which is the effective TTK.

\section{(S8) Skill Estimation}\label{s8-skill-estimation}

During my research for this project, I focused in on more proprietary
systems being used to rank players beyond ELO. During that search I
found TrueSkill, the Bayesian ranking system developed for Halo
initially. While more advanced approaches have since been created, I
settled for the original simple framework to score each player from the
playtest's overall skill.

Per-match performance scores were computed as:

\(\text{Score} = \text{Kills} - \text{Deaths} + 5 × (I_{\text{Won}})\)

Players were ranked within each match by this score and fed into
TrueSkill, which maintains a belief distribution (mean \(\mu\),
uncertainty \(\sigma\)) per player and updates it based on whether the
match results were surprising (ie: a low-rated player outscoring a
high-rated player). This converges on a statistically robust skill
rating for every unique user, with the oppurtunity for future
modifications to the performance score formula as new game modes are
developed.

This study, in particular, will extend well to future playtests where
returning playtesters' would keep their ranks. This could also be used
to test if a new mechanic is difficult for the previously skilled
players to adjust to.

\section{(S9) Playstyle Archetypes}\label{s9-playstyle-archetypes}

Next, I sought to categorize players into ``styles'' (ie: ``Passive
Gunner'', ``Speed Demon'') without manually defining the thresholds for
those categories. Internally, my brother and I always believed that
playtesters could be categorized into very specific archetypes; this
study was designed with the goal of collecting evidence of this. To test
the hypothesis that players naturally segregate into distinct
playstyles, I applied K-means clustering (\(k=3\)) to four behavioral
features:

\begin{itemize}
\tightlist
\item
  Time spent above high-speed threshold
\item
  Dashes per minute
\item
  Average absolute roll angle
\item
  Mean fight initiation distance
\end{itemize}

Features were standardized before clustering to ensure that units (ie:
degrees vs meters) did not bias the distance calculations. The algorithm
grouped players into 3 distinct clusters based on the similarity of
their feature vectors. This allows me to mathematically define
playstyles based on the data rather than human intuition.

\section{(S10) Performance
Aggregation}\label{s10-performance-aggregation}

This final study focuses on the client performance CSVs previously
dissuced.

Network performance significantly impacts gameplay fairness. Averages
can be misleading because lag spikes are rare but impactful. Therefore,
while I did aggregate all of the performance files for each player to
allow for any metric to be analyzed, I decided modeling the distribution
of packet latency would be the most helpful first analysis.

So I start by collecting raw packet latency samples from every player
across every match, filtering out invalid zero-values. Latency data is
typically right-skewed (mostly low, with a long tail of lag spikes). A
Normal distribution is a poor fit for this. Instead, I decided to fit a
Gamma Distribution to the data using maximum likelihood estimation. By
solving for the Gamma parameters (\(\alpha\), \(\beta\), location), I
generated a probability density function. This allows me to calculate
precise probabilities for ``bad experience'' thresholds, such as
determining exactly what percentage of gameplay time occurs above
\(100\) ms latency.

\section{Results}\label{results}

Since I wanted to keep this report from being too long, I'll only
show/discuss a few of the results. If you want to see any other studies,
the results are in the repo.

The study that took the most time, but had the most satisfying result
was \hyperref[s4-spatial-death-density]{s4}.
\hyperref[fig:ue5-screenshot]{Figure 1} is a screenshot inside the
Outpost map level in UE5 showing the clusters of death locations marked
by size and color.

\begin{figure}

{\centering \includegraphics[width=0.7\linewidth]{ue5-screenshot} 

}

\caption{Unreal Engine Editor Screenshot}\label{fig:ue5-screenshot}
\end{figure}

``I have already identified weak map areas, such as the Outpost
basement, where fights occurred too infrequently and with too little
positional variety. This will directly motivate changes to these map
layouts to keep all areas viable.

Another important result was from \hyperref[s7-survival-analysis]{s7},
which summarized the K/D as seen in \hyperref[fig:s7-graph]{Figure 2}.

\begin{figure}

{\centering \includegraphics[width=0.7\linewidth]{results/graphs/s7_ttk_curve_deaths} 

}

\caption{s7 TTK Graph}\label{fig:s7-graph}
\end{figure}

This establishes that the current effective K/D of Astro is three
seconds (time at which 50\% of lethal fights conclude). As we continue
to update gameplay, we can monitor if our changes are actually effecting
TTK. The more important observation from this graph, though, is how
quickly the probability of survival drops off after just a few seconds.
A major focus of development has been increasing the strategy and
opportunities to extend fights; for example, the kick mechanic grants
momentary invulnerability to help players escape direct combat. It's
clear that these prolonged fights are more rare than hoped, and we'll be
monitoring the skewness in this graph to hopefully increase not just the
median, but also the probability of sustaining active fights over longer
periods where actions are continually performed, countered, and so on.

The last result I'll focusing on was the \hyperref[fig:s3-graph]{odds
probability graph} created for the lethal items (non-lethal items like
the grapple had to be excluded for this study).

\begin{figure}

{\centering \includegraphics[width=0.7\linewidth]{results/graphs/s3_winrate_forest} 

}

\caption{s7 TTK Graph}\label{fig:s3-graph}
\end{figure}

This was the most important result to obtain, as unbalanced items have
consistently been the bane of playtests. Until now, we've only been able
to balance based on direct feedback from playtesters and our own
intuition. Now, we can actively begin the process of nerfing the
Throwing Star and buffing the Quad Cannon.

\section{Conclusions}\label{conclusions}

Overall, this project has been a successful introduction to gameplay
data analysis for Astro. As development continues, I plan to refine the
pipeline to capture more insights and connect data across different
sessions. A key focus will be linking returning players to create
longitudinal studies, which will help control for the different skill
levels of our playtesters. To support this, I intend to implement a
persistent database. Ideally, the final result will be a completely
hands-off system that runs automatically in the background after every
playtest.





\newpage
\singlespacing 
\bibliography{C:/Users/luked/Repositories/svm-r-markdown-templates/article2-example/master2.bib}

\end{document}
